O \href{http://www.freertos.org/FreeRTOS-Plus/FreeRTOS_Plus_Trace/FreeRTOS_Plus_Trace.shtml}{\tt Free\-R\-T\-O\-S+\-Trace} é uma biblioteca que pode ser integrada com o Free\-R\-T\-O\-S e que permite armazenamento de dados sobre o comportamento do programa em tempo de execução\-: chamadas de funções, carregamento de processador, interrupções e etc. Estas ficam armazenadas na memória R\-A\-M e são lidas via dump de memória (o que é feito automaticamente para o J-\/\-Link).\hypertarget{page_freertosplustrace_page_freertosplustrace_section_jlink}{}\section{Usando o J-\/\-Link}\label{page_freertosplustrace_page_freertosplustrace_section_jlink}
Após instalada, a aplicação {\bfseries Free\-R\-T\-O\-S+\-Trace} (a.\-k.\-a. Traecealyzer) pode ser executada em modo {\itshape Demo} ou {\itshape Free Version}. As duas opções possuem funcionalidades suficientes para a aplicação no projeto. Para ler os dados gravados, {\bfseries J-\/\-Link $>$ Read trace}. A região de memoria a ser lida deve englobar completamente os dados do Trace. Para tal, em modo de debug, verifica-\/se em que endereço de memória está a variável {\itshape Recoder\-Data\-Ptr}. A região setada nas configurações do Tracealyzer deve iniciar neste (ou até mesmo antes deste) endereço, e ter no mínimo o tamanho previsto nas configuracões do Trace (E\-V\-E\-N\-T\-\_\-\-B\-U\-F\-F\-E\-R\-\_\-\-S\-I\-Z\-E em trc\-Config.\-h). No exemplo, foi usado o dobro (o configurado era 4000).



\hypertarget{page_freertosplustrace_page_freertosplustrace_section_ft2232}{}\section{Usando outro J\-T\-A\-G com F\-T2232}\label{page_freertosplustrace_page_freertosplustrace_section_ft2232}
Ainda não testado. 